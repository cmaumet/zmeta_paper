% This is LLNCS.DEM the demonstration file of
% the LaTeX macro package from Springer-Verlag
% for Lecture Notes in Computer Science,
% version 2.4 for LaTeX2e as of 16. April 2010
%
\documentclass{llncs}
\usepackage{amsopn} % DeclareMathOperator
\usepackage{amsmath} % bmatrix
\usepackage{graphicx} % include graphics

\DeclareMathOperator{\Var}{Var}
\newcommand{\effectvector}{Y}
\newcommand{\effect}[1][i]{\effectvector_{#1}}
\newcommand{\vareffect}[1][i]{V_{\effect[#1]}}
\newcommand{\zeffect}[1][i]{Z_{#1}}
\newcommand{\nStudies}{k}
\newcommand{\varCombined}{\sigma^2_{C}}
\newcommand{\varBetween}{\tau^2}
\newcommand{\varWithinCommon}{\sigma^2}
\newcommand{\sampleSize}[1][i]{n_{#1}}
\newcommand{\varWithin}[1][i]{\sigma^2_{#1}}
\newcommand{\transpose}{^T}




\usepackage{makeidx}  % allows for indexgeneration
%
\begin{document}
%
\frontmatter          % for the preliminaries
%
\pagestyle{headings}  % switches on printing of running heads
% \addtocmark{Hamiltonian Mechanics} % additional mark in the TOC
%
%
\mainmatter              % start of the contributions
%
\title{Combining fMRI studies in Image-Based Meta-analysis}

%
\titlerunning{Running title}  % abbreviated title (for running head)
%                                     also used for the TOC unless
%                                     \toctitle is used
%
\author{Camille Maumet\inst{1} \and TODO pain
\and Thomas E. Nichols\inst{1,2}}
%
\authorrunning{Camille Maumet et al.} % abbreviated author list (for running head)
%
%
\institute{Warwick Manufacturing Group, The University of Warwick, Coventry, UK.\\
\and
Statistics Department, The University of Warwick, Coventry, UK.}

\maketitle              % typeset the title of the contribution

\begin{abstract}
The abstract should summarize the contents of the paper
using at least 70 and at most 150 words. It will be set in 9-point
font size and be inset 1.0 cm from the right and left margins.
There will be two blank lines before and after the Abstract. \dots
\keywords{TODO}
\end{abstract}
%
\section{Introduction}
% While most neuroimaging meta-analyses are based on peak coordinate data, the best practice method is an Intensity-Based Meta-Analysis (IBMA) that combines the effect estimates and their standard errors (E+SE's, aka COPE \& sqrt-VARCOPE) from each study [5]. Various efforts are underway to facilitate sharing of neuroimaging data to make such IBMA's possible (see, e.g. [2]), but the emphasis is usually on T-statistics and E+SE's are difficult to use in practice; for example, an analysis of E+SE's requires knowledge of the data, design and contrast scaling. However a meta-anlaysis based on T-statistic images is sub-optimal and discouraged in (non-imaging) meta-analysis [1], as the units of the meta-analysis are, say, ``BOLD significance'' instead of ``\% BOLD change''.


% Given a set of $k$ studies, we can face different configuration of data sharing, and hence for each study $i$ having available for meta-analysis:
% \begin{enumerate}
% 	\item the contrast estimates $\effect$ and contrast variance estimates $\vareffect$.
% 	\item the contrast estimates $\effect$.	
% 	\item the standardized statistical maps $\zeffect$.		
% \end{enumerate}

% Depending on how much is shared different strategies can be used to combine the available results into a meta-analysis.

% Here we compare the use of IMBA using only T-statistics to use of E+SE's. Using 21 studies of pain in control subjects, we compare the best-practice analysis to two approaches using only T-statistics, Stouffer's [7] and weighted Z-score [4], the latter accounting for differing study sample size.

% As an alternative to parametric procedures, non-parametric statistics rely on a very small set of assumptions. 

\section{Methods}
\subsection{Theory}
Given a set of $k$ studies, we denote for each study $i$: its contrast estimate by $\effect$, its contrast variance estimate by $\vareffect$, its standardized statistical map by $\zeffect$ and its sample size by $n_i$.

\begin{table}[h]
\begin{center}
\begin{tabular}{ccc}
						& Statistic			& Disbribution under H$_0$ \\
\hline						
FFX GLM 		& $ \displaystyle \frac{1}{ \sqrt{\sum_{i=1}^\nStudies 1/\vareffect }} \sum_{i=1}^\nStudies \frac{\effect}{\vareffect}$ & $\mathcal{T}_{ (\sum_{i=1}^\nStudies n_i - 1) - 1}$ \\
MFX GLM 		& $ \displaystyle \frac{1}{ \sqrt{\sum_{i=1}^\nStudies 1/ (\vareffect + \hat\varBetween) }} \sum_{i=1}^\nStudies \frac{\effect}{\vareffect + \hat\varBetween}$ & $\mathcal{T}_{\nStudies - 1}$ \\
RFX GLM 		& $ \displaystyle \frac{1}{\widehat\varCombined / \sqrt{\nStudies}} \sum_{i=1}^\nStudies \frac{\effect}{\nStudies}$ & $\mathcal{T}_{\nStudies - 1}$ \\
Contrast Permutation	& $ \displaystyle \frac{1}{\widehat\varCombined / \sqrt{\nStudies}} \sum_{i=1}^\nStudies \frac{\effect}{\nStudies}$ & \parbox{5cm}{Determined through permutations with sign switching.} \\
Fisher's	& $\displaystyle -2 \sum_{i=1}^{\nStudies} \ln( \Phi(-\zeffect) ) )$ & $\chi^2_{(2\nStudies)}$\\
Stouffer's& $\displaystyle  \frac{\sum_{i=1}^\nStudies \zeffect}{\sqrt{\nStudies}}$ & $\mathcal{N}(0,1)$ \\
Optimally weighted-Z	& $\displaystyle  \frac{\sum_{i=1}^\nStudies  \sqrt{n_i} \zeffect}{\sqrt{\sum_{i=1}^\nStudies n_i}}$ & $\mathcal{N}(0,1)$ \\
Stouffer's MFX& $\displaystyle  \frac{\sum_{i=1}^\nStudies \zeffect}{\sqrt{\nStudies} \hat \sigma}$ & $\mathcal{T}_{\nStudies-1}$ \\
Z Permutation	& $\displaystyle  \frac{\sum_{i=1}^\nStudies \zeffect}{\sqrt{\nStudies}}$ & \parbox{5cm}{Determined through permutations with sign switching.} \\
\hline 

\end{tabular}
\end{center}
\caption{Statistics for one-sample meta-analysis tests and distributions under the null hypothesis.}
\label{stat_table}
\end{table}	

\paragraph{Combining contrast estimates and their standard error}

The gold standard approach to combine contrast estimates and their standard errors is to input them into a GLM~\cite{Cummings2004}, creating effectively the third-level of a hierarchical model (level 1: subject; level 2: study; level 3: meta-analysis). The general formulation is:

\begin{equation}
	\effectvector = X \beta + \epsilon
	\label{eq_meta_GLM}
\end{equation}

where $\beta$ is the meta-analytic parameter to be estimated, $Y = [\effect[1] \ldots \effect[\nStudies] ]\transpose$ is the vector of contrast estimates and $\epsilon \sim \mathcal{N}(0,W)$ is the residual term. Eq.~\eqref{eq_meta_GLM} can be solved by weighted least square giving:

% TODO add contrast

\begin{eqnarray}
	\hat \beta  &=& (X\transpose W X)^{-1} X\transpose W \effectvector \\
	\Var(\hat \beta)  &=& (X\transpose W X)^{-1}
	\label{eq_WLS}
\end{eqnarray}

In a random-effects model, we have $W = diag( \varWithin[1] + \varBetween \ldots \varWithin[\nStudies] + \varBetween )$ where $\varBetween$ denotes the between-studies variance. Approximating $\varWithin$ by $\vareffect[i]$ and given $\hat\varBetween$ an estimate of $\varBetween$ we obtain the statistics detailed in table~\ref{stat_table} for a one sample test. This reference approach will be referred to as \textbf{Mixed-effects (MFX) GLM}.

In a fixed-effects model (i.e. assuming no between-study variances), we have $W = diag( \varWithin[1] \ldots \varWithin[\nStudies] )$ where $\varWithin$ denotes the contrast variance for study $i$. This approach will be referred to as \textbf{Fixed-effects (FFX) GLM}.


\paragraph{Combining contrast estimates}
In the absence of standard error, the contrast estimates $\effect$ can be combined by assuming that the within-study variance $\varWithin$ is roughly constant ($\varWithin \simeq \sigma^2 \; \forall \, 1 \le i \le \nStudies $) or negligible by comparison to the between-study variance($\varWithin \ll \varBetween \; \forall \, 1 \le i \le \nStudies $). Then $W = diag( \varCombined \ldots \varCombined )$ where $\varCombined$ is the combined within and between-subject variance such as $\varCombined \simeq \varBetween$ or $\varCombined \simeq \varBetween + \varWithinCommon$. Under these assumptions, eq.~\eqref{eq_meta_GLM} can be solved by ordinary least square giving:

\begin{eqnarray}
	\hat \beta  &=& (X\transpose X)^{-1} X\transpose \effectvector \\
	\Var(\hat \beta)  &=& (X\transpose W X)^{-1}
	\label{eq_OLS}
\end{eqnarray}

Given $\hat\varCombined$ an estimate of $\varCombined$ we obtain the statistics presented in table~\ref{stat_table} for one sample tests. This approach will be referred to \textbf{Random-effects (RFX) GLM} in the following.

As an alternative to parametric approaches, non-parametric statistics~\cite{Holmes1996,Nichols2002} can be computed by comparing the RFX GLM T-statistic to the distribution obtained by permuting the sign of each sample included in the analysis. This approach will be referred to as \textbf{Contrast permutation}.

\paragraph{Combining standardised statistics} 
In the presence of standardised statistical estimates, \textbf{Fisher's} meta-analysis provide a statistic to combine the associated p-values~\cite{Fisher1932}. \textbf{Stouffer's} approach combines directly the standardised statistic~\cite{Stouffer1949}. In \cite{Zaykin2011} following \cite{Liptak1958}, the author proposed a weighted method that weights each study's $\zeffect$ by the square root of its sample size [3,7]. This approach will be referred to as \textbf{Optimally weighted-Z}. All these meta-analytic statistics assumes fixed-effects (no between-study variance) and are suited only for one-sample tests. The corresponding statistics are presented in table~\ref{stat_table}.

As suggested in~\cite{Salimi-khorshidi2009}, to get a kind of MFX with Stouffer's approach, the standardised statistical estimates $\zeffect$ can be combined in an OLS analysis. The corresponding estimate, referred as \textbf{Stouffer's MFX} is also provided in~\ref{stat_table}

As an alternative to parametric approaches, a non-parametric distribution~\cite{Holmes1996,Nichols2002} can be estimated by permutation on the $\zeffect$'s. This approach will be referred to as \textbf{Z permutation}.

\subsection{Experiments}

\subsubsection{Simulations}
To verify the validity of each estimator under the null hypothesis we estimated the false positive rate at $p<0.05$ uncorrected. For each meta-analysis, we simulated a contrast estimate and a variance estimates such as:
\begin{eqnarray}
	\effect &\sim& \mathcal{N}(0, \frac{\varWithin}{\sampleSize}+\varBetween) \\
	\vareffect &\sim& \frac{\varWithin}{\sampleSize-1} \chi^2_{(\sampleSize-1)}%
\end{eqnarray}
where $\varWithin \in [1/2, 1, 2, 4]$ is the within-study variance, $\varBetween \in [0, 1]$ is the between-study variance (fixed-effects if $\varBetween$ is $0$, random-effects otherwise). We simulated different number of studies per meta-analysis: $\nStudies \in [5, 10, 25, 50]$ and the number of subjects per studies $\sampleSize$ was selected such as we would have varying number of subjects per studies in given meta-analysis across the common range of subjects involved in neuroimaging studies. In each simulated meta-analysis we simulated one study with exactly 20, 25, 10 and 50 subjects. For the remaining studies the number of subjects were drawn from uniform distributions a quarter from $\mathcal{U}(11,20)$, a quarter from $\mathcal{U}(26,50)$ and the remaining from $\mathcal{U}(21,25)$. A total of 32 parameter sets (4 $\varWithin$ x 2 $\varBetween$ x 4 $\nStudies$) was therefore tested, 71 repeats with 5041 samples per repeats were simulated.


\subsubsection{Real data}

We first compared the Z-scores obtained by the three approaches using a Bland-Altman plot. Then, as results are usually presented as a thresholded map, we computed the dice similarity score between thresholded maps obtained with Stouffer's and weighted-Z FFX with FLAME FFX for three (uncorrected) thresholds: p < 0.001, 0.01 and 0.05. Finally, as results are best reported using a multiple comparison correction, we defined ground truth activations as the FLAME FFX analysis FDR-corrected at a threshold of p<0.05 and plotted Receiver-Operating-Characteristics (ROC) curves of Stouffer's and weighted-Z FFX.

All plots were generated using ggplot~\cite{ggplot}.

\section{Results}


\subsection{Simulations}
Fig.~\ref{fig_fpr_all} displays the false positive rate at $p<0.05$ obtained for the eight estimators over all set of parameters in the absence and presence of random-effects. From this graph, it is clear that the fixed-effects meta-analytic summary statistics, i.e.\ Fisher's, Stouffer's and weighted-z estimates are overly liberal in the presence of random-effects. As expected the original Fisher's approach is the most invalid. Surprisingly, FFX GLM is also invalid under fixed-effects, maybe suggesting inaccurate degrees of freedoms (here set to $(\sum_{i=1}^\nStudies n_i - 1) - 1)$). Stouffer's MFX, GLM RFX and permutations of $\effect$'s or $\zeffect$'s provide valid estimates. The permutation estimates present the largest sampling variance.

\begin{figure}[ht]
	\centering
	\includegraphics[width=\linewidth]{./Rplot_FPR_all.pdf}
	\caption{False positive rates of the meta-analytic estimators under the null hypothesis for $p<0.05$.}
	\label{fig_fpr_all}
\end{figure}

The impact of the number of studies involved in the meta-analysis and of the size of the within-study variance are investigated in fig.~\ref{fig_fpr_valid}. The permutation estimates appears conservative ($\text{FPR}\simeq 0.03$) when 5 studies are involved. All approaches perform equally as soon as 10 or more studies are included in the meta-analysis. 
\begin{figure}[ht]
	\centering
	\includegraphics[width=\linewidth]{./Rplot_valids.pdf}
	\caption{False positive rates of the valid random-effects meta-analytic estimators under the null hypothesis for $p<0.05$ as a function of the number of studies and the within-study variance.}
	\label{fig_fpr_valid}
\end{figure}


\subsection{Real data}

\begin{figure}[ht]
	\centering
	\includegraphics[width=0.7\linewidth]{./Rplot_ratio_variances.pdf}
	\caption{Histogram of the between-study variance to the sum of the between-subject variance and the mean within-study variance.}
	\label{fig_realdata_variances}
\end{figure}

The histogram of the ratio of between-subject variance onto total variance is displayed in fig.~\ref{fig_realdata_variances}. From this graph it is clear that for most of the voxels the estimated between-study variance is greater than the within-study variance. We can therefore suppose the presence of random-effects (non zero between-study variance) in this dataset.

\begin{figure}[ht]
	\centering
	\includegraphics[width=\linewidth]{./Rplot_realdata_ffx.pdf}
	\caption{Difference between the z-score estimated from each FFX meta-analytic approach and the reference z-score from MFX GLM as a function of reference z-score.}
	\label{fig_realdata_ffx}
\end{figure}


Fig.~\ref{fig_realdata_ffx} and fig.~\ref{fig_realdata_rfx} plots the difference between the z-score estimated by each meta-analytic approach and the reference z-score computed with MFX GLM for FFX and RFX approaches respectively. All FFX statistics provide overly optimistic z-estimate suggesting, again, that random-effects are indeed present in the studied dataset.


\begin{figure}[ht]
	\centering
	\includegraphics[width=\linewidth]{./Rplot_realdata_rfx.pdf}
	\caption{Difference between the z-score estimated from each RFX meta-analytic approach and the reference z-score from MFX GLM as a function of reference z-score.}
	\label{fig_realdata_rfx}
\end{figure}


Among the RFX meta-analytic approaches presented in fig.~\ref{fig_realdata_rfx}, GLM RFX and contrast permutations provide z-scores estimate that are equal or smaller than the reference. Z permutation provides slightly larger z-scores between 1 and 3 (reference p-values between 0.16 and 0.0013) but is mostly in agreement with the reference z-scores. On the other hand, Stouffer's MFX is more liberal than the reference for z-score ranging from 3 to 5 (reference p-values between 0.0013 and 2.9e-07) and more stringent for z-scores smaller than 5.

% Fig. 1 shows the Bland-Altman plots comparing Z-scores from the Stouffer's and weighted-Z methods each compared with the ground truth Z-scores. Overall, both approaches present the same pattern of overestimation of the Z-scores. The weighted-Z approach provides a somewhat more condensed pattern suggesting a closer match to the ground truth.
% The dice similarity score for uncorrected p-values of 0.001, 0.01 and 0.05 were 0.84, 0.87 and 0.89 respectively for Stouffer's method and 0.86, 0.88 and 0.90 for the weighted Z-score, showing again slightly better results for the weighted-Z approach. These scores are notably higher (dice similarity scores range from 0 to 1) than the scores obtained with coordinate-based meta-analyses (around 0.5, [5]).
% Finally the ROC curves displayed in figure 2 for a ground truth obtained with an FDR corrected threshold p<0.05 demonstrate again a slight advantage of weighted-Z FFX over Stouffer's FFX.

% Dice among valids
% \begin{enumerate}
% \item StouffersMFX: 0.9454
% \item PermutZ: 0.9450
% \item GLMRFX: 0.8994
% \item PermutCon: 0.8991
% \end{enumerate}

% \begin{enumerate}
% \item WeightedZ: 0.9244
% \item Stouffers: 0.9184
% \item GLMFFX: 0.8972
% \item fishers: 0.8382
% \end{enumerate}

% AUC between 0 and 0.1
% among valids
% \begin{enumerate}
% \item StouffersMFX: 0.8924
% \item PermutZ: 0.8919
% \item GLMRFX: 0.7809
% \item PermutCon: 0.7815
% \end{enumerate}

% \begin{enumerate}
% \item  WeightedZ: 0.8293
% \item  Stouffers: 0.8619
% \item  fishers: 0.6329
% \item  GLMFFX: 0.6111
% \end{enumerate}
    

\section{Conclusion}
We have compared eight meta-analytic statistic in the context of one-sample test. Through simulations, we outlined the invalidity of standard FFX approaches in the presence of random-effects. In a real dataset of 21 studies of pain, we outline the presence of random-effects advocating for the use of RFX meta-analytic statistics. When contrast estimates only are available, the RFX procedure was valid. This is in line with previous results on within-group one-sample t-tests studies~\cite{Mumford2009}. When only standardised estimates are available, permutation is the preferred option as the one providing the most faithful results. Further investigations are needed in order to investigate the behaviour of these estimators in other configurations, including meta-analyses focusing on between-study differences.


\section{Acknowledgements}
We gratefully acknowledge the use of this data from the Tracey pain group, FMRIB, Oxford.

%
% ---- Bibliography ----
\bibliographystyle{plain}
\bibliography{miccai2014}

% %
% \begin{thebibliography}{5}
% %
% \bibitem {clar:eke}
% Clarke, F., Ekeland, I.:
% Nonlinear oscillations and
% boundary-value problems for Hamiltonian systems.
% Arch. Rat. Mech. Anal. 78, 315--333 (1982)

% \bibitem {clar:eke:2}
% Clarke, F., Ekeland, I.:
% Solutions p\'{e}riodiques, du
% p\'{e}riode donn\'{e}e, des \'{e}quations hamiltoniennes.
% Note CRAS Paris 287, 1013--1015 (1978)

% \bibitem {mich:tar}
% Michalek, R., Tarantello, G.:
% Subharmonic solutions with prescribed minimal
% period for nonautonomous Hamiltonian systems.
% J. Diff. Eq. 72, 28--55 (1988)

% \bibitem {tar}
% Tarantello, G.:
% Subharmonic solutions for Hamiltonian
% systems via a $\bbbz_{p}$ pseudoindex theory.
% Annali di Matematica Pura (to appear)

% \bibitem {rab}
% Rabinowitz, P.:
% On subharmonic solutions of a Hamiltonian system.
% Comm. Pure Appl. Math. 33, 609--633 (1980)

% \end{thebibliography}


\end{document}
